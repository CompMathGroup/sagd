\documentclass[12pt,a4paper]{article}
%\usepackage[cp1251]{inputenc}
\setlength{\voffset}{-1in}
\setlength{\topmargin}{20mm}
\setlength{\headheight}{0mm}
\setlength{\headsep}{0mm}
\setlength{\hoffset}{-1in}
\setlength{\oddsidemargin}{25mm}
\setlength{\marginparwidth}{0mm}
\setlength{\marginparsep}{0mm}
\setlength{\textheight}{247mm}
\setlength{\textwidth}{165mm}
\usepackage{indentfirst}
\usepackage{setspace}
\usepackage{tabularx}
\usepackage{graphicx}
\usepackage{epstopdf}
\usepackage[utf8]{inputenc}
\usepackage[T2A]{fontenc}
\usepackage{amsmath,amssymb}
\usepackage[english, russian]{babel}
\usepackage[hidelinks]{hyperref}
\newcommand{\pd}[2]{\frac{\partial #1}{\partial #2}}

\linespread{1.5}
%\captionsetup{width=\textwidth}

%\title{Численное моделирование парогравитационной технологии добычи высоковязких нефтей}
%\author{Фирсов Егор}
\begin{document}
%\maketitle
{
	
\thispagestyle{empty}
\begin{center}
Министерство образования и науки Российской Федерации
\\
Федеральное государственное автономное образовательное учреждение\\[-6pt] высшего профессионального образования\\[-6pt]
«Московский физико-технический институт\\[-6pt]
(государственный университет)»
\\
Факультет аэрофизики и~космических исследований\\[-6pt]
Кафедра вычислительной математики\\
\end{center}

\vspace{20mm}

\begin{center}
{\large {\bf Численное моделирование парогравитационной технологии добычи высоковязких нефтей\\[8mm] }} 
Выпускная квалификационная работа \\
(бакалаврская работа)\\
Направление подготовки: 03.03.01 Прикладные математика и физика
\end{center}

\vspace{20mm}

\begin{flushleft}
\begin{tabularx}{\textwidth}{lcl}
Выполнил: \\
Студент 131 группы  & \rule{5cm}{1pt} & Фирсов Егор Игоревич \\[5mm]

Научный руководитель:\\
к.ф.-м.н., доцент & \rule{5cm}{1pt} & Скалько Юрий Иванович \\[5mm]
Научный консультант:\\
д.ф.-м.н.              & \rule{5cm}{1pt} & Колдоба Александр Васильевич \\
\end{tabularx}
\end{flushleft}

\vfill

\begin{center}
 Москва 2015
\end{center}
}

\newpage
\tableofcontents
\newpage
\section{Введение}


Значительная доля нефтяных запасов России приходится на высоковязкие нефти и битумы. Поэтому проблема добычи высоковязких нефтей и битумов крайне актуальна. Их добыча сложна из-за большой вязкости ($\mu > 30$ Па с). 

Существуют технологии помогающие при добыче такой нефти, один из них это подвод тепла к области, в том числе с помощью технологии парогравитационного дренажа. Эта технология состоит из двух этапов. Первый этап — подогрев области с помощью циркулирующего внутри труб пара. Второй - закачка горячего пара в слой.

\begin{figure}[h!]
\begin{center}
\includegraphics[width=14cm]{./other/SAGD_2.jpg}
\caption{Схема парагравитационного дренажа}
\end{center}
\end{figure}

\subsection {Постановка задачи}

В данной работе рассмотрен первый этап парогравитационного дренажа, то есть подогрев области. В отличии от традиционной задачи фильтрации здесь скелет не является монолитной структурой, а подвижен. Смоделирована одномерная полубесконечная область, расположенная вертикально. Она заполнена двумя подвижными фазами песком и нефтью. Холодная нефть имеет большую вязкость и практически не фильтруется. Нижняя граница области имеет большую температуру. Благодаря конвекции и теплопроводности область начинает прогреваться, вязкость нефти уменьшается, и песок начинает оседать под действием силы тяжести.

В начале получены уравнения, необходимые для решения поставленной задачи. Далее задача разбита на несколько подзадач, задачу изотермической фильтрации, теплопроводности, изотермической фильтрации с предельной насыщенностью, и наконец решена задачу неизотермической задачи с предельной насыщенностью. При решении каждой подзадачи были построены вычислительные алгоритмы и разработаны програмные комплексы, реализующие их. Также для изотермической задачи было построено аналитическое решение.



%\newpage
\subsection{Обозначения}
Индекс a - значит l и s - флюид и твердая фаза
\begin{itemize}
\item $\theta_a $ --- Объемная доля a-ой фазы
\item $\rho_a$ --- Плотность а-ой фазы ($[\rho_a] = \text{Кг/м}^3$)
\item $e_a$ --- удельная энергия a-ой фазы
\item $T $ --- Температура ($[T] =\text{К}$)
\item $\lambda$ --- Коэффициент теплопроводности среды ($[\lambda] = \text{Дж/(м с К)}$)
\item $c_a$ --- Теплоемкость а-ой фазы ($[c_a] = \text{Дж/К}$)
\item $V_a$ --- Скорость а-ой фазы ($[V_a] = \text{м/с} $)
\item $W$ --- Скорость фильтрации ($[W] = \text{м/с} $)
\item $h_a$ --- удельная энтальпия
\item $\psi$ - отношение удельных объемов двух фаз($\psi = \frac{\theta_l}{\theta_s} $)
\end{itemize}


\newpage
\section{Математическая модель}
\subsection{Уравнения модели}
Выпишем уравнения необходимые для решения задачи. Это уравнение непрерывности, уравнение сохранения энергии и определяющее соотношение в виде закона Дарси.  $z$ --- вертикальная координата, $t$ --- время. В уравнениях присутствуют 2 индкса --- $ _l $ и $ _s$, означающие, что данная величина относится к жидкой фазе (нефти) или твердой фазе(песку) соответственно. $\theta$ --- объемная доля соответсвующей фазы, $ V $ --- скорость.

Уравнения непрерывности, при условии что источников вещества нет, выглядит следующим образом:
\begin{equation}
\pd{\rho_l\theta_l}{t} + \pd{V_l\theta_l\rho_l}{z} =0
\label{eq:contin_l}
\end{equation}
\begin{equation}
\pd{\rho_s\theta_s}{t} + \pd{V_s\theta_s\rho_l}{z} =0
\label{eq:contin_s}
\end{equation}

Сумма объемных долей жидкой и твердой фазы равна единице.
\begin{equation}
\theta_l + \theta_s = 1
\label{eq:satur}
\end{equation}

Считаем что источников энергии нет, тогда уравнение сохранения энергии принимает следующий вид:
\begin{equation}
\begin{aligned}
&\pd{E}{t} + \pd{Q}{z} =0 \\
&E=\theta_l\rho_l e_l + \theta_s\rho_s e_s \\
&Q=\theta_l\rho_l h_l V_l + \theta_s\rho_s h_s V_s - \lambda \pd{T}{z}\\
\label{eq:conserv}
\end{aligned}
\end{equation}

%\begin{equation}
%\pd{\theta_l\rho_l e_l + \theta_s\rho_s e_s }{t} + \pd{ \theta_l\rho_l h_l + \theta_s\rho_s h_s - \lambda \pd{T}{z} }{z} = 0
%\end{equation}
Считаем, что твердая фаза несжимаема

\begin{equation}
\begin{aligned}
%\rho_l = const \\ 
\rho_s = const \\
\label{rho_const}
\end{aligned}
\end{equation}

Тогда из уравнения \eqref{eq:contin_s}, с учетом несжимаемости флюида получаем
\begin{equation}
\pd{\theta_s}{t} + \pd{\theta_s V_s}{z} = 0
\label{fluid}
\end{equation}

Учитывая \eqref{fluid} перейдем к Лагранжевой координате $dx$ связанной с движением твердой фазы.
\begin{equation}
dx = \theta_s dz - \theta_s V_s dt
\label{dx_dz}
\end{equation}

Чтобы переписать наши уравнения в новой Лагранжевой координате посмотрим как связаны дифференциирование по $ z $ и по $ x $
\begin{equation}
\begin{aligned}
&\left(\pd{f}{t}\right)_z = \pd{(f , z)}{(t , z)}\\
&\partial(f , z) = \frac{1}{\theta_s}\partial(f , x) + V_s \partial(f , t)\\
&\pd{(f , z)}{(t , z)} = \frac{1}{\theta_s}\pd{(f , x)}{(t , z)} + V_s\pd{(f , t)}{(t , z)}\\
\end{aligned}
\label{help1}
\end{equation}

Для $ f = t $ 
$$
\partial(t , z) = \frac{1}{\theta_s} \partial(t , x) + V_s \partial(t , t) = \frac{1}{\theta_s} \partial(t , x)
$$

Из \eqref{help1}, используя предыдущее равенство получаем
$$
\left(\pd{f}{t}\right)_z = \pd{(f , x)}{(t , x)} + V_s \theta_s\pd{(f ,t)}{(t , x)}
$$

Далее получаем
\begin{equation}
\begin{aligned}
&\left(\pd{f}{t}\right)_z = \left(\pd{f}{t}\right)_x - V_s\theta_s\left(\pd{f}{x}\right)_t\\
&\left(\pd{f}{z}\right)_t = \theta_s\left(\pd{f}{x}\right)_t\\
\end{aligned}
\label{z_x}
\end{equation}

Применим полученные равенства к уравнению непрерывности для твердой фазы \eqref{fluid}. Переходим к новой перемнной.
$$
\pd{\theta_s}{t} - V_s\theta_s\pd{\theta_s}{x} + \theta_s\pd{V_s\theta_s}{x} =0
$$

Расписывая третий член, и сокращая подобные члены, получаем
$$
\pd{V_s}{x} + \frac{1}{\theta_s^2}\pd{\theta_s}{t} = 0
$$

Отсюда
\begin{equation}
\pd{V_s}{x} = \pd{}{t}\frac{1}{\theta_s}
\label{V_s}
\end{equation}

\subsection{Уравнение непрерывности в лагранжевых координатах}
Теперь используя полученные равенства перепишем уравнение непрерывности для флюида в новой переменной. Для начала перепишем его добавив и вычтя член $V_s\pd{\rho_l\theta_l}{z} $
$$
\pd{\rho_l\theta_l}{t} + V_s\pd{\rho_l\theta_l}{z} - V_s\pd{\rho_l\theta_l}{z} + \pd{V_l\theta_l\rho_l}{z} =0
$$

Сгруппировав члены и используя \eqref{z_x}, получаем:
$$
\left(\pd{\rho_l\theta_l}{t}\right)_x - V_s\theta_s \left(\pd{\rho_l\theta_l}{x}\right)_t + \theta_s\left(\pd{\rho_l\theta_l(V_l - V_s)}{x}\right)_t + \rho_l\theta_l\theta_s\left(\pd{V_s}{x}\right)_t + V_s\theta_s \left(\pd{\rho_l\theta_l}{x}\right)_t = 0
$$

Перепишем это, используя \eqref{V_s} и поделив все на $\theta_s$
$$
\frac{1}{\theta_s}\pd{\rho_l\theta_l}{t} + \pd{\rho_l\theta_l(V_l-V_s)}{x} + \rho_l\theta_l\pd{}{t}\frac{1}{\theta_s} = 0
$$

Сгруппировав первый и последний члены выпишем
\begin{equation}
\pd{}{t}\frac{\rho_l\theta_l}{\theta_s} + \pd{}{x}\rho_l\theta_l(V_l - V_s) = 0
\label{filtr_compr_1}
\end{equation}

Введем скорость фильтрации $W$ как
\begin{equation}
W = \theta_l (V_l - V_s )
\label{W_filtr}
\end{equation}

С учетом этого и в предположении, что флюид не сжимаемый, получаем уравнение непрерывности в лагранжевых координатах
\begin{equation}
\pd{}{t}\frac{\theta_l}{\theta_s} + \pd{}{x}W = 0
\label{filtr_compr}
\end{equation}

\subsection{Уравнение сохранения энергии в лагранжевых координатах}
Перепишем уравнение сохранения энергии в лагранжевой переменной. Используем \eqref{eq:conserv} и \eqref{z_x}
$$
\pd{E}{t} - V_s\theta_s\pd{E}{x} + \theta_s\pd{Q}{x} = 0
$$

Отсюда
$$
\frac{1}{\theta_s}\pd{E}{t} + \pd{}{x}(Q-V_s E) + E\pd{V_s}{x} = 0
$$

Используем \eqref{V_s}, группируя члены получаем
\begin{equation}
\pd{}{t}\frac{E}{\theta_s} + \pd{}{x}(Q - V_s E) = 0
\end{equation}

$$
Q-V_s E = \theta_l\rho_l e_l V_l + \theta_s\rho_s e_s V_s + \theta_l P + \theta_s P -V_s(\rho_l\theta_l e_l +\rho_s\theta_s e_s) - \lambda\theta_s\pd{T}{x}
$$

Сокращаем подобные члены, учитываем $W = \theta_l (V_l - V_s )$, получаем уравнение сохранения энергии в новых переменных
\begin{equation}
\pd{}{t}\frac{E}{\theta_s} + \pd{}{x}(\rho_l e_l W - \lambda\theta_s\pd{T}{x}) = 0
\label{eq:conserv_new}
\end{equation}

Это уравнение наряду с уравнением \eqref{filtr_compr} являются основными для решения поставленной задачи. Кроме них нужно ещё одно уравнение для вычисления скорости фильтрации. 

\subsection{Закон Дарси в лагранжевых координатах}
Выпишем закон Дарси.
\begin{equation}
W= -\frac{K k(\theta)}{\eta}\left(\pd{p}{z} - \rho_l g \right)
\label{Darsi}
\end{equation}

Здесь $ K $ --- абсолютная проницаемость(она константа, далее она будет опускатся и просто входить в проницаемость k). $k(\theta)$ --- проницаемость, функция от объемных долей жидкой и твердой фаз. $p$ --- давление. 

Исключим из него давление, используя условие равновесия
$$
\pd{p}{z} = (\rho_l \theta_l + \rho_s \theta_s)g
$$

Получим следствие из закона Дарси:
\begin{equation}
W= -\theta_s \frac{k(\theta)}{\eta}(\rho_l-\rho_s)g
\label{Darsi_new}
\end{equation}

Введем новую переменную равную отношению объемных долей : 
$$
\psi = \frac{\theta_l}{\theta_s} 
$$

Все величины $\theta_l $, $\theta_s $, $\psi $ - выражаются друг через друга, поэтому возможно использование любой из них. Для удобства будем использовать $\psi$

Для проницаемости используется комбинация формул. В пределе малой объемной доли твердой фазы, когда взаимодействием между частицами можно принебречь, используется следствие формулы Стокса 
$$
k = 2 K\frac{\psi^2}{(1+\psi)^2} 
$$

В пределе большой объемной доли твердой фазы, когда учитыватся взаимодействие между частицами используется формула Кармана-Козени 
$$
k = K\frac{\psi^3}{(1+\psi)^2}
$$

$\theta_s $ внесена в проницаемость $k$ и формулы переписаны в переменной $\psi$ . 

Комбинируем эти формулы (берем среднее гармоническое). Получаем:

\begin{equation}
k = K \frac{4\psi^3}{(1+\psi)^2(2+\psi)}
\label{permability}
\end{equation}

\begin{figure}[h!]
\begin{center}
\includegraphics[width=9cm]{./other/K_diplom.png}
\caption{График функции проницаемости}
\end{center}
\end{figure}
%--------------------------------------------------------------------------------------------------------------
\newpage
\section{Изотермическая фильтрация}
\subsection{Уравнения}
Для решения задачи изотермической фильтрации достаточно уравнения непрерывности, закона Дарси и выражения для проницаемости. Выпишем их:
\begin{equation}
\begin{aligned}
&\pd{\psi}{t} + \pd{}{x}W = 0\\
&W= \frac{k(\psi)}{\eta}(\rho_s-\rho_l)g\\
&k = K \frac{4\psi^3}{(1+\psi)^2(2+\psi)}\\
\label{isotermal}
\end{aligned}
\end{equation}
\subsection{Численный метод}
Запишем разностную схему
\begin{itemize}
\item Сначала вычисляем проницаемость на всей области.
\begin{equation}
k_i = k(\psi_i) , \ i = 0,  \dots ,M
\label{perm_razn}
\end{equation}

\item Далее вычисляем скорости фильтрации на границах всех ячеек. Используем противопоточную аппроксимацию. То есть сносим нужные значения из ячейки расположеной против потока. Считаем, что скорость фильтрации направлена всегда вверх, поэтому на границе $i + \frac12$ сносим проницаемость из ячейки $i$.
\begin{equation}
W_{i+\frac12} = \frac{k(\psi_i)}{\eta}(\rho_s-\rho_l)g \ , \ i = 0,  \dots ,M-1
\label{Darsi_razn}
\end{equation}

\item Наконец переходим на новый слой по времени, вычисляем отношение насыщенностей в каждой ячейке
\begin{equation}
\psi_i^{n+1} = \psi_i^n - \tau\frac{W_{i+1/2} - W_{i-1/2}}{h} \ , \ i = 1,  \dots ,M
\label{filtr_razn}
\end{equation}
\end{itemize}

\subsection{Аналитическое решение}
В \eqref{isotermal}, подставляя выражение для скорости фильтрации и проницаемости и расписывая диффиринииирование по $x$ получаем:
\begin{equation}
\pd{\psi}{t} - a K \pd{k}{\psi} \pd{\psi}{x} = 0,\  \text{Где}\ a = \frac{8(\rho_l-\rho_s)g}{\eta}
\label{characteristics}
\end{equation}

Получили гиперболическое уравнение, которое решается методом характеристик. 
$$
\xi = \frac{x}{t} = a K \pd{k}{\psi} =   a K \frac{2\psi^3 + 3\psi^2}{(1 + \psi)^3(2+\psi)^2}
$$

Для начального условия $\psi_0 = 5$ и граничного $\psi_1 = 0$, что означет, что в области находится смесь жидкости и песка, а снизу непротекаемая граница из песка характеристики выглядят следующим образом.
\begin{figure}[h!]
\begin{center}
\includegraphics[width=12cm]{./other/charact_corr.eps}
\caption{Характеристики}
\end{center}
\end{figure}

\subsection{Результаты численных рачетов}
На графике --- распределение объемной доли флюида в конкретный момент времени. График построен в лагранжевых координатах. Синим пунктиром изображено аналитическое решение. Красной сплошной линией --- разностное решение. На графке видно 3 различных области --- слева центрированая волна где уплотняется песок, затем скачок и стационарная область. Как видно по графику решение разностной задачи неплохо аппроксимирует аналитическое решение, немного размывается рядом со скачком, что является последствием использования схемы первого порядка.
\begin{figure}[h!]
\begin{center}
\includegraphics[width=16cm]{./other/new_Filtration_3.jpg}
\caption{Распределение объемной доли флюида}
\end{center}
\end{figure}

%--------------------------------------------------------------------------------------------------------------
\newpage
\section{Теплопроводность}
\subsection{Математическая модель}
Решим следующую задачу. 
Пусть объемная доля твердой фазы во всей рассматриваемой области достигла предельного значения. И на нижней границе задана скорость фильтрации равная нулю. Тогда скорость фильтрации во всей области равна нулю. Температура границы больше температуры области. Уравнение сохранения энергии \eqref{eq:conserv_new} принимает вид:
$$
(c_l \psi + c_s)\pd{T}{t} =-\theta_s q
$$
$$
q = -\lambda\pd{T}{x}
$$

\subsection{Численный метод}
Для решения используется явная схема.
\begin{equation}
\begin{aligned}
&q_{m+1/2} = - \lambda\frac{T_{m+1} - T_m}{h}\\
&T_m^{m+1} = - \tau \theta_s\frac{q_{m+1/2} - q_{m-1/2}}{c_l \psi + c_s} + T_m^n\\
\label{termal_razn}
\end{aligned}
\end{equation}
\subsection{Аналитическое решение}
Перепишем уравнение сохранения энергии в виде уравнения теплопроводности и решим его аналитически:
\begin{equation}
\pd{T}{t} = \varkappa\pd{^2 T}{x^2},\ \text{где } \varkappa = \frac{\lambda \theta_s}{\rho_l c_l \psi + \rho_s c_s}
\end{equation}

Его решение имеет вид:
\begin{equation}
T(x,t) = T_1-T_0 - T_0 \operatorname{erfc} \left(\frac{x}{\sqrt{4\varkappa t}}\right)	
\end{equation}
где erfc(x) - Дополнительная функция ошибок:
$$
erfc(x) = \int_{x}^{+\infty}e^{\eta^2}d\eta
$$
\subsection{Результаты численных расчетов}
Зеленым изображено численне решение, черным пунктиром --- аналитическое. Из графика видно что численное решение хорошо сходится с аналитическим.
\begin{figure}[h!]
\begin{center}
\includegraphics[width=155mm]{./other/Temp_Analyt.png}
\caption{Распределение Температуры}
\end{center}
\end{figure}
%--------------------------------------------------------------------------------------------------------------
\newpage
\section{Изотермическая фильтрация с предельной насыщенностью}
\subsection{Математическая модель}
Вообще объемная доля твердой фазы не может достигнуть единицы. В любом случае останется пространство между частицами, заполненное флюидом. Получаем два случая, первый когда объемная доля твердой фазы меньше предельной. В этом случае верно следствие закона Дарси, приведенное в задаче изотермической фильтрации. Второй случай когда объемная доля твердой фазы досигает предельной. Тогда уравнения принимают вид стандартной однофазной фильтрации. Для песчинок случайного размера предельная объемная доля $\theta_s = 0.64 $ \cite{Dobr_phys02}.

\begin{equation}
\begin{cases}
\theta_s < \theta_s^*,\  W = \frac{k(\psi)}{\eta} (\rho_s - \rho_l) g\\
\theta_s = \theta_s^*,\  \pd{W}{x} = 0\\
\end{cases}
\end{equation}

Уравнение непрерывности и выражение для проницаемости остаются теми же, что и для задачи изотермической фильтрации\eqref{isotermal}.

\subsection{Численный метод}
Вычисляем проницаемость на всей области.
\begin{equation}
k_i = k(\psi_i) , \ i = 0,  \dots ,M
\label{perm_razn2}
\end{equation}

Далее вычисляем скорость фильтрации на границе очередной ячейки. Как и ранее используется противопоточная аппроксимация.
\begin{equation}
W_{i+1/2} = \frac{k(\psi_i)}{\eta}(\rho_s-\rho_l)g
\label{Darsi_razn2}
\end{equation}

Вычисляем отношение насыщенностей в очередной ячейке. 
\begin{equation}
\psi_i^{n+1} = \psi_i^n - \tau\frac{W_{i+1/2} - W_{i-1/2}}{h}
\label{filtr_razn2}
\end{equation}
Существование предельной объемной доли тверой фазы, больше которой оно не может вырасти, означает существование предельного отношения объемных долей, меньше которого оно не может упасть.

Если $\psi$ больше предельного, переходим на следующую ячейку.

Если же наоборот оно получилось меньше предельного, приравниваем его предельному и перерасчитываем скорость фильтрации, используя уравнение непрерывности \eqref{filtr_razn2} с известным отношением объемных долей равным предельному.

\begin{equation}
\begin{aligned} 
&\psi_{i}^{n+1} = \psi^* \\
&W_{i+1/2} = \frac{(\psi_i^n - \psi^*)h}{\tau} + W_{i-1/2} \\
\end{aligned}
\label{corr_razn}
\end{equation}
\subsection{Результаты численных расчетов}

\newpage
\section{Неизотермическая фильтрация с предельной насыщенностью}
\subsection{Математическая модель}

Выпишем ещё раз все уравнения.
$$
\pd{\psi}{t} + \pd{}{x}W = 0
$$

Ранее коэффициент вязкости $\eta$ был константой. Теперь же он зависит от температуры по экспоненциальному закону.
\begin{equation}
\eta = e^{\alpha(T - T_{0})}
\label{viscos}
\end{equation}

Этот закон частный случай закона Фогеля. Он хорошо аппроксимирует поведение вязкости нефти на небольшом промежутке температур.

Закон Дарси имеет такой же вид как и в прошлом разделе.
$$
\begin{cases}
\theta_s < \theta_s^*,\  W = \frac{k(\psi)}{\eta(T)} (\rho_s - \rho_l) g\\
\theta_s = \theta_s^*,\  \pd{W}{x} = 0\\
\end{cases}
$$

Для расчета температуры понадобится уравнение сохранения энергии \eqref{eq:conserv_new}. Распишем его:	
$$
\pd{(e_l\psi+e_s)}{t} + \pd{}{x}(e_l W - \lambda\theta_s\pd{T}{x}) = 0
$$
где $ e_l = \rho_l c_l T,\ e_s = \rho_s c_s T $ .



\subsection{Численный метод}
Для вычисления потоков тепла между ячейками и температуры на границах, как и для скорости фильтрациии используем противопоточную аппроксимацию.
\begin{enumerate}
\item Вычисляем вязкость на всей области : 
$$
\eta_m = e^{-\alpha(T_m^n - T_{0})},\ m = 0, \dots ,M
$$
\item Вычисляем относительную проницаемость:
$$
k_m = k(\psi_m) , \ m = 0,  \dots ,M
$$
\item Вычисляем потоки тепла между ячейками:
$$
q_{m+\frac12} = - \lambda (\theta_{s})_m^{n} \frac{T_{m+1}^n - T_m^n}{h} 
$$

Теперь проходим не весь временной слой сразу, а идем по одной ячейке, что бы корректировать скорость фильтрации в случае достижения предельной объемной доли.
\item Вычисляем скорость фильтрации на очередной границе ячеек:
$$
W_{m+\frac12} = -\frac{k_{m+\frac12}}{(\eta)_{m+\frac12}}(\rho_s - \rho_l)g
$$
\item  вычсляем отношение объемных долей:
$$
\psi_m^{n+1} = \psi_m^n - \tau \frac{W_{m+\frac12} - W_{m-\frac12}}{h}
$$
\item Если отношение объемных долей упало меньше предельной, приравниваем его предельному и корректируем скорость филтьрации:
$$
\begin{aligned}
&\psi_{i}^{n+1} = \psi^* \\
&W_{i+1/2} = \frac{(\psi_i^n - \psi^*)h}{\tau} + W_{i-1/2} \\
\end{aligned}
$$
\item Вычисляем температуру на новом слое. 
\begin{multline}
 T_m^{n+1} =  \left(\frac{1}{\psi_m^{n+1} c_l + c_s}\right)\left( (\psi_m^n  c_l + c_s)
- \tau\frac{q_{m+\frac12} - q_{m-\frac12}}{h} - \tau\frac{c_l T_{m+\frac12}^n W_{m+\frac12} - c_l T_{m-\frac12} W_{m-\frac12}}{h}\right)
\end{multline}
\end{enumerate}

Уравнение непрерывности является гиперболическим. Для устойчивости необходимо соблюдение условия Куранта. Расчитаем шаг по времени необходимый для устойчивости схемы для уравнения непрерывности:
$$
\frac{\tau a_{max}}{h} < 1,\ \ a_{max} = \left|\pd{W}{\psi}\right|_{max}
$$
$$
\tau_1 = \frac{h}{2 a_{max}},\ \  a_{max} = 8 K (\rho_s - \rho_l)g\left|\frac{2\psi^3 + 3\psi^2}{\eta(1 + \psi)^3(2+\psi)^2}\right|_{max}
$$
%Уравнение сохранения энергии - Параболическое...
\subsection{Результаты численных расчетов}

\begin{figure}[h!]
\begin{center}
\includegraphics[width=150mm]{./other/new_heat+noheat_2.png}
\caption{Распределение объемной доли флюида}
\end{center}
\end{figure}

\begin{figure}[h!]
\begin{center}
\includegraphics[width=150mm]{./other/new_heat_Temp.png}
\caption{Распределение температуры}
\end{center}
\end{figure}

\begin{figure}[h!]
\begin{center}
\includegraphics[width=150mm]{./other/new_heat_Visc.png}
\caption{Распределение вязкости}
\end{center}
\end{figure}
\section{Заключение}
Основные результаты:
\begin{itemize}
\item Построена математическая модель задачи неизтермической фильтрации с предельной насыщенностью
\item Разработан вычислительный алгоритм моделирующий эту задачу.
\item Были построены аналитические решения для задач теплопроводности и изотермической фильтрации для тестирования програмной реализации.
\item Проведены расчеты.
\item Далее планируется смоделировать второй этап парогазового воздействия --- непосредственно саму закачку пара, а также расширить задачу на двумерный случай. 
\end{itemize}
\newpage
\bibliographystyle{utf8gost71u} 
\bibliography{literature} 

\end{document}