\documentclass[12pt]{article}
%\usepackage[cp1251]{inputenc}
\usepackage[utf8]{inputenc}
\usepackage[T2A]{fontenc}
\usepackage{amsmath,amssymb}
\usepackage[english, russian]{babel}
\newcommand{\pd}[2]{\frac{\partial #1}{\partial #2}}
\title{Численное моделирование парогравитационной технологии добычи высоковязких нефтей}
\author{Фирсов Егор}
\begin{document}
\maketitle
\newpage
\section*{Обозначения}

\section*{Уравнения}

Уравнение непрерывности. Считаем, что источников нет
\begin{equation}
\pd{\rho_l\theta_l}{t} + \pd{V_l\theta_l\rho_l}{z} =0
\label{eq:contin_l}
\end{equation}
\begin{equation}
\pd{\rho_s\theta_s}{t} + \pd{V_s\theta_s\rho_l}{z} =0
\label{eq:contin_s}
\end{equation}

Уравнение сохранения энергии. Считаем, что источников нет
\begin{equation}
\begin{aligned}
&\pd{E}{t} + \pd{Q}{z} =0 \\
&E=\theta_l\rho_l e_l + \theta_s\rho_s e_s \\
&Q=\theta_l\rho_l h_l + \theta_s\rho_s h_s - \lambda \pd{T}{z}\\
\label{eq:conserv}
\end{aligned}
\end{equation}

%\begin{equation}
%\pd{\theta_l\rho_l e_l + \theta_s\rho_s e_s }{t} + \pd{ \theta_l\rho_l h_l + \theta_s\rho_s h_s - \lambda \pd{T}{z} }{z} = 0
%\end{equation}
Считаем, что скелет несжимаемый

\begin{equation}
\begin{aligned}
%\rho_l = const \\ 
\rho_s = const \\
\label{rho_const}
\end{aligned}
\end{equation}

Тогда из уравнения \eqref{eq:contin_s}, с учетом несжимаемости флюида получаем
\begin{equation}
\pd{\theta_s}{t} = - \pd{\theta_s V_s}{z}
\label{fluid}
\end{equation}

Перейдем к новой переменной $dx$
\begin{equation}
dx = \theta_s dz - \theta_s V_s dt
\label{dx_dz}
\end{equation}

Отсюда, для как-либо функции f 
\begin{equation}
\begin{aligned}
&\left(\pd{f}{t}\right)_z = \pd{(f , z)}{(t , z)}\\
&\partial(f , z) = \frac{1}{\theta_s}\partial(f , x) + V_s \partial(f , t)\\
&\pd{(f , z)}{(t , z)} = \frac{1}{\theta_s}\pd{(f , x)}{(t , z)} + V_s\pd{(f , t)}{(t , z)}\\
\end{aligned}
\label{help1}
\end{equation}

Для $ f = t $ 
$$
\partial(t , z) = \frac{1}{\theta_s} \partial(t , x) + V_s \partial(t , t) = \frac{1}{\theta_s} \partial(t , x)
$$

Из \eqref{help1}, используя предыдущее равенство получаем
$$
\left(\pd{f}{t}\right)_z = \pd{(f , x)}{(t , x)} + V_s \theta_s\pd{(f ,t)}{(t , x)}
$$

Далее получаем
\begin{equation}
\begin{aligned}
&\left(\pd{f}{t}\right)_z = \left(\pd{f}{t}\right)_x - V_s\theta_s\left(\pd{f}{x}\right)_t\\
&\left(\pd{f}{z}\right)_t = \theta_s\left(\pd{f}{x}\right)_t\\
\end{aligned}
\label{z_x}
\end{equation}

Применим полученные равенства к уравнению непрерывности для скелета \eqref{fluid}. Переходим к новой перемнной.
$$
\pd{\theta_s}{t} - V_s\theta_s\pd{\theta_s}{x} + \theta_s\pd{V_s\theta_s}{x} =0
$$

Расписывая третий член, и сокращая подобные члены, получаем
$$
\pd{V_s}{x} + \frac{1}{\theta_s^2}\pd{\theta_s}{t}
$$

Отсюда
\begin{equation}
\pd{V_s}{x} = \pd{}{t}\frac{1}{\theta_s}
\label{V_s}
\end{equation}

Теперь используя полученные равенства перепишем уравнение непрерывности для флюида в новой переменной. Для начала перепишем его добавив и вычтя $V_s\pd{\rho_l\theta_l}{z} $
$$
\pd{\rho_l\theta_l}{t} + V_s\pd{\rho_l\theta_l}{z} - V_s\pd{\rho_l\theta_l}{z} + \pd{V_l\theta_l\rho_l}{z} =0
$$

Сгруппировав члены и используя \eqref{z_x}, получаем:
$$
\left(\pd{\rho_l\theta_l}{t}\right)_x - V_s\theta_s \left(\pd{\rho_l\theta_l}{x}\right)_t + \theta_s\left(\pd{\rho_l\theta_l(V_l - V_s)}{x}\right)_t + \rho_l\theta_l\theta_s\left(\pd{V_s}{x}\right)_t + V_s\theta_s \left(\pd{\rho_l\theta_l}{x}\right)_t = 0
$$

Перепишем это, используя \eqref{V_s} и поделив все на $\theta_s$
$$
\frac{1}{\theta_s}\pd{\rho_l\theta_l}{t} + \pd{\rho_l\theta_l(V_l-V_s)}{x} + \rho_l\theta_l\pd{}{t}\frac{1}{\theta_s} = 0
$$

Сгруппировав первый и последний члены выпишем
\begin{equation}
\pd{}{t}\frac{\rho_l\theta_l}{\theta_s} + \pd{}{x}\rho_l\theta_l(V_l - V_s) = 0
\label{filtr_compr}
\end{equation}

Обозначим 
\begin{equation}
W = \theta_l (V_l - V_s )
\label{W_filtr}
\end{equation}

С учетом этого и в предположении, что флюид не сжимаемый, получаем
\begin{equation}
\pd{}{t}\frac{\theta_l}{\theta_s} + \pd{}{x}W = 0
\label{filtr_compr}
\end{equation}

Перепишем уравнение сохранения энергии в новой переменной. Используем \eqref{eq:conserv} и \eqref{z_x}
$$
\pd{E}{t} - V_s\theta_s\pd{E}{x} + \theta_s\pd{Q}{x}
$$

Отсюда
$$
\frac{1}{\theta_s}\pd{E}{t} + \pd{}{x}(Q-V_s E) + E\pd{V_s}{x} = 0
$$

Используем \eqref{V_s} и группируя члены получаем
\begin{equation}
\pd{}{t}\frac{E}{\theta_s} + \pd{}{x}(Q - V_s E) = 0
\end{equation}

$$
Q-V_s E = \theta_l\rho_l e_l V_l + \theta_s\rho_s e_s V_s + \theta_l P + \theta_s P -V_s(\rho_l\theta_l e_l +\rho_s\theta_s e_s) - \lambda\theta_s\pd{T}{x}
$$

Сокращаем подобные члены, учитываем $W = \theta_l (V_l - V_s )$, получаем уравнение сохранения энергии в новых переменных
\begin{equation}
\pd{}{t}\frac{E}{\theta_s} + \pd{}{x}(\rho_l e_l W - \lambda\theta_s\pd{T}{x}) = 0
\label{eq:conserv_new}
\end{equation}

Выпишем закон Дарси
\begin{equation}
W= -\frac{k}{\eta_l}\left(\pd{p}{z} + (\rho_l-\rho_s)g \right)
\label{Darsi}
\end{equation}
%--------------------------------------------------------------------------------------------------------------
\newpage
\section*{Подзадача 1. Фильтрация}

Решаем задачу фильтрации в следующих предположениях:
\begin{itemize}
\item Скелет несжимаемый $\rho_s = const $
\item Флюид несжимаемый $\rho_l = const $
\item Проницаемость имеет квадратичную зависимость от насыщенности флюида
\begin{equation}
k=\theta_l^2
\label{perm}
\end{equation}
\end{itemize}
Уравнение непрерывности приобретает вид
$$
\pd{}{t}\frac{\theta_l}{\theta_s} + \pd{}{x}W = 0
$$
Закон Дарси:
$$
W= -\frac{k}{\eta_l}(\rho_l-\rho_s)g
$$

Выпишем разностную схему

Сначала вычисляем проницаемость на всей области.
\begin{equation}
k_i = (\theta_{l,i})^2 , \ i = 0,  \dots ,M-1
\label{perm_razn}
\end{equation}

Далее вычисляем скорости фильтрации на границах всех ячеек
\begin{equation}
W_{i+1/2} = \frac{K k(\theta_i)}{\eta_l}(\rho_s-\rho_l)g \ , \ i = 0,  \dots ,M-1
\label{Darsi_razn}
\end{equation}

Наконец переходим на новый слой по времени, вычисляем отношение насыщеносей в каждой ячейке
\begin{equation}
\left(\frac{\theta_l}{\theta_s}\right)_i^{n+1} = \left(\frac{\theta_l}{\theta_s}\right)_i^n - \tau\frac{W_{i+1/2} - W_{i-1/2}}{h} \ , \ i = 1,  \dots ,M-1
\label{filtr_razn}
\end{equation}

%--------------------------------------------------------------------------------------------------------------
\newpage
\section*{Подзадача 2. Теплопроводность}
Решаем задачу теплопроводности в следующих предположениях:
\begin{itemize}
\item Одна среда
\item Среда несжимаемая $\rho = const$
\item Источников энергии нет
\end{itemize}

Тогда уравнение сохранения энергии примет следующий вид
\begin{equation}
\rho c \pd{T}{t} = - \pd{q}{z}
\label{termal_1}
\end{equation}

\begin{equation}
q = - \lambda\pd{T}{z}
\label{termal_2}
\end{equation}

Запишем разностную схему
\begin{equation}
q_{m+1/2} = - \lambda\frac{T_{m+1} - T_m}{h}
\label{termal_razn_1}
\end{equation}

\begin{equation}
T_m^{m+1} = - \tau\frac{q_{m+1/2} - q_{m-1/2}}{h\rho c} + \rho c T_m^n
\label{termal_razn_2}
\end{equation}

%--------------------------------------------------------------------------------------------------------------
\newpage
\section*{Подзадача 3. Фильтрация с конечной насыщенностью}
Введем новую переменную равную отношению насыщенностей : 
$$
\psi = \frac{\theta_l}{\theta_s} 
$$

Считаем, что насыщенность скелета имеет предел, когда он уже не может больше уплотнятся. То есть существует такое $ \psi^* $ - критическое, что $\psi $ не может стать меньше чем $\psi^*$. 

Тогда разностная схема изменится. 
Сначала вычисляем проницаемость на всей области.
\begin{equation}
k_i = (\theta_{l,i})^2 , \ i = 0,  \dots ,M-1
\label{perm_razn2}
\end{equation}

Далее вычисляем скорость фильтрации на границе очередной ячейки.
\begin{equation}
\psi_i^{n+1} = \psi_i^n - \tau\frac{W_{i+1/2} - W_{i-1/2}}{h}
\label{filtr_razn2}
\end{equation}

Вычисляем отношение насыщенностей в очередной ячейке. 
\begin{equation}
W_{i+1/2} = \frac{K k(\theta_{l,i})}{\eta_l}(\rho_s-\rho_l)g
\label{Darsi_razn2}
\end{equation}
Если оно больше критического переходим на следующую ячейку.

Если же наоборот оно получилось меньше критического, приравниваем его критическому и перерасчитываем скорость фильтрации используя уравнение \eqref{Darsi_razn2} с известным отношением насыщенностей равным критическому.

\begin{equation}
\begin{aligned}
&\psi_{i}^{n+1} = \psi^* \\
&W_{i+1/2} = \frac{(\psi_i^n - \psi^*)h}{\tau} + W_{i-1/2} \\
\end{aligned}
\label{corr_razn}
\end{equation}


\end{document}